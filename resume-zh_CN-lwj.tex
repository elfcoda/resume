% !TEX TS-program = xelatex
% !TEX encoding = UTF-8 Unicode
% !Mode:: "TeX:UTF-8"

\documentclass{resume}
\usepackage{zh_CN-Adobefonts_external} % Simplified Chinese Support using external fonts (./fonts/zh_CN-Adobe/)
% \usepackage{NotoSansSC_external}
% \usepackage{NotoSerifCJKsc_external}
% \usepackage{zh_CN-Adobefonts_internal} % Simplified Chinese Support using system fonts
\usepackage{linespacing_fix} % disable extra space before next section
\usepackage{cite}

\begin{document}
\pagenumbering{gobble} % suppress displaying page number

\name{卢文杰}

\basicInfo{
  \email{707935952@qq.com} \textperiodcentered\ 
  \phone{(+86) 155-749-89048} \textperiodcentered\ 
  \linkedin[]{}}
 
\section{\faGraduationCap\  教育背景}
\datedsubsection{\textbf{湖南大学}, 长沙}{2013 -- 2017}
\textit{学士}\ 软件工程

\section{\faCogs\ IT 技能}
% increase linespacing [parsep=0.5ex]
\begin{itemize}[parsep=0.5ex]
  \item 编程语言: C++(C++11以上特性,能在项目中使用基本的元编程技巧)  >  Python  > >  Haskell
  \item 前端技能: Html, CSS, Javascript
  \item 工具: Git,Vim
  \item 能够从零开始设计并实现简单的编程语言,并利用LLVM实现优化的后端代码
  \item 了解分布式数据库TiDB的设计与实现
  \item 了解CPU的基本工作原理和优化技巧,对从上层(应用层)到底层(CPU级别)的优化都有一定的了解
  \item 由于严格的Code Review,对代码质量有较高的追求
  \item 喜欢做有趣的有挑战的事情,对静态分析,编译方向,安全方向,性能优化都感兴趣,有学习的热情
\end{itemize}

\section{\faUsers\ 实习/工作}
\datedsubsection{\textbf{腾讯 - IEG光子} \ 深圳}{2016年7月 -- 2016年9月}
\role{实习}{游戏预研组员工}
游戏服务器后端开发
\begin{itemize}
  \item 一个关于protobuf数据转换的小工具,将pb文件转换成C++文件输出数据
\end{itemize}

\datedsubsection{\textbf{虎牙直播} \ 广州}{2017年7月 -- 2018年8月}
\role{正式员工}{直播技术部}
后台开发
\begin{itemize}
  \item 直播排行榜。ToC业务的关键技术:Redis、多线程、缓存、高并发、协程、Tars微服务框架
  \item 服务通过用户打赏的通知接口,来更新每个主播的总排名,以及对这个主播打赏最多的前3名观众。用户进入直播间会拉取排行榜接口,如果当前主播的榜单未改变,拉取之前的缓存。
  \item 设计与客户端通信的协议,设计分布式缓存,通过内存缓z冲区来提高服务的可用性,牺牲强一致性,在适当的业务情况下牺牲最终一致性
  \item 通过LRU、栈上内存交换,读写锁,内存粒度控制等技术提高服务性能,通过Redis缓存,多线程+epoll+协程池提高业务吞吐
\end{itemize}

\datedsubsection{\textbf{腾讯 - CSIG腾讯企点} \ 上海}{2018年9月 -- 2020年3月}
\role{正式员工}{后台开发组}
后台开发
\begin{itemize}
  \item 利用协程的栈机器特性优雅地实现服务日志改造
  \item 维护并开发满意度服务,基于kafka消息队列实现高可用满意度服务,从不同的服务收敛满意度业务到单独的服务中,提高系统的可维护性。
\end{itemize}

\datedsubsection{\textbf{微软 - CMD-MMD} \ 苏州}{2020年12月 -- 至今}
\role{正式员工}{后端开发}
后台开发
\begin{itemize}
  \item 与西雅图团队合作,参与设计并实现UpdateManagement服务,利用Azure上Cosmos全球高可用缓存数据库,FunctionApp等组件,做前端请求到后端服务的转发与缓存,避免流量过大导致请求失败超时,多地部署在北美,欧洲等数据中心。
\end{itemize}

\section{\faCode\ 项目经历}
\datedsubsection{\textbf{个人网站}}{2017 年 -- 2018 年}
\role{Html, CSS3, Javascript, MySQL, Django, Python}{个人项目}
\begin{onehalfspacing}
个人网站, http://flowerdance.me/
\begin{itemize}
  \item 有一定的产品sense:弹幕,寻路弹幕,友链弹幕,代码弹幕(将会基于parser实现),有较好用户体验
  \item Django框架开发,实现博客的增删改查功能
\end{itemize}
\end{onehalfspacing}

\datedsubsection{\textbf{jhin(烬)语言编译器}}{2020 年7月 -- 2020 年10月}
\role{C++17, Python3}{个人项目}
\begin{onehalfspacing}
静态强类型语言, https://github.com/elfcoda/jhin,关注架构的设计,关注代码的可维护性。代码API的通用性与易用性,注重下游程序员的编程体验。
\begin{itemize}
  \item 实现词法分析器,将定义的正则由NFA转换为DFA,再分析出词法元素,易于调试
  \item 实现LALR自下而上语法分析器,并解析语法定义文件,生成相关代码;语法易于修改和扩展;能够检测出语法冲突并打印出相关调试信息
  \item 由语法分析器生成解析树和AST
  \item 遍历AST,做简单的一阶类型系统分析,能够检测出类型错误
  \item 遍历AST,使用LLVM生成Mac平台优化的汇编代码,并能够无缝链接C函数库
  \item 提供对应简单的VsCode语法插件:https://github.com/elfcoda/Ashe-lang-support,实现语法高亮等功能
  \item 工具:Python工具,生成语法分析需要的相关定义;生成解析树和AST需要的数据,用于图像化树结构,便于调试和开发;通用的相关comm库,如日志系统,能方便打印出各种类型数据(包括递归容器)
  \item 缺陷:没有词法定义文件,相关代码不是生成的,所以修改的时候为了代码一致性需要改多个地方。这一点在语法分析阶段得到了优化
\end{itemize}
\end{onehalfspacing}

% Reference Test
%\datedsubsection{\textbf{Paper Title\cite{zaharia2012resilient}}}{May. 2015}
%An xxx optimized for xxx\cite{verma2015large}
%\begin{itemize}
%  \item main contribution
%\end{itemize}

\section{\faInfo\ 其他}
% increase linespacing [parsep=0.5ex]
\begin{itemize}[parsep=0.5ex]
  \item \faHome\ 技术博客: http://flowerdance.me/
  \item \faGithubAlt\ GitHub: https://github.com/elfcoda
\end{itemize}

%% Reference
%\newpage
%\bibliographystyle{IEEETran}
%\bibliography{mycite}
\end{document}
