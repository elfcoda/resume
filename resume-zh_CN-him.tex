% !TEX TS-program = xelatex
% !TEX encoding = UTF-8 Unicode
% !Mode:: "TeX:UTF-8"

\documentclass{resume}
\usepackage{zh_CN-Adobefonts_external} % Simplified Chinese Support using external fonts (./fonts/zh_CN-Adobe/)
%\usepackage{zh_CN-Adobefonts_internal} % Simplified Chinese Support using system fonts
\usepackage{linespacing_fix} % disable extra space before next section
\usepackage{cite}

\begin{document}
\pagenumbering{gobble} % suppress displaying page number

\name{刘丰恺}

\basicInfo{
  \email{lfk\_dsk@hotmail.com} \textperiodcentered\ 
  \phone{(+86) 183-4087-9772} \textperiodcentered\ 
  \github[lfkdsk]{https://github.com/lfkdsk}}

\section{\faGraduationCap\  教育背景}
\datedsubsection{\textbf{大连理工大学}, 大连}{Sep 2014 -- 至今}
\textit{在读学士}\ 软件工程, 预计 2018 年 6 月毕业

\section{\faUsers\ 主要经历}
\datedsubsection{\textbf{EL-Engine 表达式编译引擎} {\textperiodcentered\ 阿里巴巴业务平台事业部}}{June 2017 -- 至今}
\role{项目开发者和主要维护者}

系统获取请求需要依赖表达式判断进行业务路由分发,因而系统需要一个能够进行高性能的表达式计算引擎。在实习期间独立完成了 EL-Engine 的引擎设计实现工作,独立实现了针对 Expression Language 级语言的解释运行和 Runtime 编译及优化

\begin{itemize}
  \item EL-Engine 的表达式计算相比项目之前依赖的表达式引擎增速 20\%
  \item 针对线上用例进行优化后达到 数量级 的增速,在 AST Runtime 优化 、常量节点折叠、Lexer \& Parser 、Runtime Compile、Template Generate  等可优化部分均有大量的优化实践和效果
  \item 表达式解析引擎在系统中占较为重要的作用,核心位置高频执行,日 pv 过亿
\end{itemize}

\datedsubsection{\textbf{学伴} {\textperiodcentered\ OurEDA创新团队}}{June 2016 -- 至今}
\role{3.0 版本开发者和维护者}

\begin{itemize}
  \item 第一次接手拥有大型用户量的应用,主要负责Android端的开发,开始对开发始于Android 开发萌芽时期的旧版本进行部分重构与维护。之后在开发3.0版本的过程中引入并在其中学到了依赖注入、响应式编程,RESTful API等业界先进的思想和工具,为日后的持续迭代和维护提供了保障,完成后客户端代码达到了7w+。
  \item 3.0版本上线后,用户量提升到1.5w人左右,日活量有5k左右。
\end{itemize}

\datedsubsection{\textbf{智能空调扇} {\textperiodcentered\ OurEDA创新团队}}{Mar 2015 -- May 2015}
\role{Android 开发工程师}

\begin{itemize}
  \item 智能空调扇是通过 智能手机控制的空调扇,手机应用和空调扇均由我们自主设计,能实现用智能手机控制风扇的温度,风向,风速和使用自然风算法进行智能调节,学习了很多嵌入式和单片机的知识。
  \item 和专门负责硬件的同学进行交互学到了关于蓝牙、WiFi连接的知识,阅读文献实现调节算法。
\end{itemize}


\datedsubsection{\textbf{云手帐} {\textperiodcentered\ OurEDA创新团队}}{Sep 2015 -- Nov 2015}
\role{Android 开发工程师}

\begin{itemize}
  \item 云手帐是一款记录性的 APP 支持笔迹、文本(支持Markdown)、代码(coder隐藏功能)的同步上传和记录。负责整个项目的设计和 Android 端的实现。
  \item 使用了自己之前在 Github 开发的开源项目 JustWeTools 项目。
\end{itemize}

\section{\faGithubSquare\ 开源项目}
\datedsubsection{\textbf{JustWeEngine}}{}
\begin{itemize}
  \item 项目地址:{\github[JustWeEngine]{https://github.com/lfkdsk/JustWeEngine}} {\textperiodcentered\ Star: 685}
  \item 简介:基于 Android 平台的、面向原生的、2D游戏开发框架,目前稳定版本已推进到v1.13,新版本正在开发中,下一版本将支持内嵌的 DSL 游戏脚本(目前解释器已开发完毕)、Canvas控件布局(对Canvas画出的控件使用布局)。JustWeEngine 可以大大的简化游戏开发中的学习成本,利于一些小型游戏程序的编写。并且各种的相关的工具包也在持续更新中,为其提供更多强大的功能。
  \item 使用情况:上一个小版本的下载量超过三千,用户有一定的活跃度,目前也在不断积累中,已有数个成型的项目接入了 JustWeEngine 用户项目页面。
\end{itemize}

\datedsubsection{\textbf{HobbyScript}}{}
\begin{itemize}
  \item 项目地址:{\github[HobbyScript]{https://github.com/lfkdsk/HobbyScript}} {\textperiodcentered\ Star: 70}
  \item 能运行在 JVM 上的一门自制语言,兼具动态语言和静态语言的部分特性,目前其解释器已经通过 binding 和 JustWeEngine 进行连接作为一种的一门 DSL 用来编写游戏脚本。
  \item AST 动态化,语法的扩展非常容易。
\end{itemize}

\datedsubsection{\textbf{JustWeTools}}{}
\begin{itemize}
  \item 项目地址:{\github[JustWeTools]{https://github.com/lfkdsk/JustWeTools}} {\textperiodcentered\ Star: 596}
  \item 自定义控件的制品集合,仅包含不常见的开源控件,包括PaintView画图工具,CodeView代码编辑 ,MarkDownView支持MarkDown语法的文字渲染器,VerTextView支持竖行排版/下划线的TextView等控件,为有需求的用户提供了很大的便利和参考。
\end{itemize}


\section{\faCogs\ 技术文章}
% increase linespacing [parsep=0.5ex]
\begin{itemize}[parsep=0.5ex]
  \item {\github[知乎专栏:SICP的魔法 https://goo.gl/ar5b1Y]{https://zhuanlan.zhihu.com/lfkdsk}}
  \item {\github[源码阅读系列:EventBus https://goo.gl/bvgLmJ]{https://lfkdsk.github.io/2016/12/22/read-eventbus-source-code/}}
  \item {\github[源码阅读系列:Rxjava2 https://goo.gl/i7YLim]{http://lfkdsk.github.io/2017/09/04/read-source-code-rxjava2/}}
  \item {\github[译文:选择使用正确的 Markdown Parser https://goo.gl/qvAHkT]{https://lfkdsk.github.io/2017/02/24/translate-md/}}
\end{itemize}

\section{\faInfo\ 其他}
% increase linespacing [parsep=0.5ex]
\begin{itemize}[parsep=0.5ex]
  \item 技术博客: https://lfkdsk.github.io
  \item GitHub: https://github.com/lfkdsk
  \item 语言: 英语 - 熟练(CET-6)
\end{itemize}

\section{致谢}

  感谢您花时间阅读我的简历,期待能有机会和您共事。

%% Reference
%\newpage
%\bibliographystyle{IEEETran}
%\bibliography{mycite}
\end{document}
